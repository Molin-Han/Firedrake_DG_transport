\documentclass{article}
\usepackage{amsmath,amssymb,stmaryrd}
\def\MM#1{\boldsymbol{#1}}
\newcommand{\pp}[2]{\frac{\partial #1}{\partial #2}} 
\newcommand{\dede}[2]{\frac{\delta #1}{\delta #2}}
\newcommand{\dd}[2]{\frac{\diff#1}{\diff#2}}
\newcommand{\dt}[1]{\diff\!#1}
\def\MM#1{\boldsymbol{#1}}
\DeclareMathOperator{\diff}{d}
\DeclareMathOperator{\DIV}{DIV}
\DeclareMathOperator{\D}{D}
\bibliographystyle{plain}
\newcommand{\vecx}[1]{\MM{#1}}
\newtheorem{definition}{Definition}
\newcommand{\code}[1]{{\ttfamily #1}} 
%uncomment \solnsfalse to remove solution set
\newif\ifsolns
%\solnstrue
\solnsfalse

\ifsolns
% with solutions
\newcommand{\soln}[1]{\newline \noindent {\bfseries Solution:} {\itshape #1}}
\else
% without solutions
\newcommand{\soln}[1]{}
\fi
\usepackage[margin=2cm]{geometry}

\usepackage{fancybox}
\begin{document}
\title{Notes on conservative, bounded, advection schemes}
\author{Colin Cotter}
\maketitle

As described in \cite{thuburn1996multidimensional}, the conservation
of a tracer $q$ can be most easily derived from the conservative form 
of the advection equations
\begin{equation}
(\rho q)_t + \nabla\cdot(\MM{u}\rho q)=0,
\end{equation}
where the density satisfies the continuity equation
\begin{equation}
\rho_t + \nabla\cdot(\MM{u}\rho) = 0.
\end{equation}
On the other hand, the $L_\infty$ bounds for $q$ are most easily derived
from the characteristic form
\begin{equation}
q_t + \MM{u}\cdot\nabla q = 0.
\end{equation}

\section{Continuity equation}
The first step is that we need a positive scheme for $\rho$,
\emph{i.e.}, $\rho(x,t)>0$ if $\rho(x,0)>0$. 

The time-continuous discontinuous Galerkin (DG) scheme for $\rho \in
DG(n)$ is defined on an element $e$ with boundary $\partial e$ by
\begin{equation}
\dd{}{t}\int_e \phi \rho \diff x - \int_e \nabla \phi \cdot\MM{u} \rho\diff x
+ \int_{\partial e} \phi \MM{u}\cdot\MM{n} \tilde{\rho} \diff s = 0,
\quad \forall \phi \in DG(n), 
\end{equation}
where $\MM{n}$ is the normal to the boundary, and $\tilde{\rho}$ is
the value of $\rho$ on the upwind side of $\partial e$.

Here we consider the forward Euler time discretisation, given by
\begin{equation}
\int_e \phi \rho^{n+1} \diff x 
= \int_e \phi \rho^n \diff x 
+ \Delta t \int_e \nabla \phi \cdot\MM{u} \rho^n\diff x
- \Delta t\int_{\partial e} \phi \MM{u}\cdot\MM{n} \tilde{\rho}^n \diff s = 0,
\quad \forall \phi \in DG(n).
\end{equation}
Forward Euler steps can be concatenated in SSPRK schemes to obtain 
higher-order time discretisations.

The above linear scheme is not positivity-preserving. To obtain a
positivity-preserving scheme, we take the following steps:

\begin{enumerate}
\item A pre-processing limiter applied to $\rho$, such as the
  Barth-Jespersen limiter or the Kuzmin vertex-based limiter, that
  sets bounds $(\rho_{\min},\rho_{\max})$ on each element $e$, based
  on the mean value $\overline{\rho}^n_e$, defined by
\begin{equation}
\overline{\rho}^n = \frac{1}{|e|}\int_e \rho^n \diff x,
\end{equation}
and the value in selected surrounding elements. The limiter is then
defined by 
\begin{equation}
\Pi_1\rho^n|_e = \overline{\rho}^n + \alpha (\rho^n - \overline{\rho}^n),
\end{equation}
with $0\leq \alpha \leq 1$ chosen as the maximum value such that
$\Pi\rho^n$ lies within the bounds in element $e$. Note that such a
limiter does not alter the element integral of $\rho$, and so does not
violate conservation.

If the element mean values of $\rho^n$ are all positive, then so are
the bounds, and therefore $\Pi\rho^n$ is positive everywhere.
\item A second pre-processing limiter $\Pi_2$ to guarantee that the
  element means $\overline{\rho}^{n+1}$ will all be positive following the
  Euler step. The goal of this section is to introduce such a limiter.
\item The application of the forward Euler step,
\begin{equation}
\label{eq:forward euler rho}
\int_e \phi \rho^{n+1} \diff x 
= \int_e \phi \rho^* \diff x 
+ \Delta t \int_e \nabla \phi \cdot\MM{u} \rho^*\diff x
- \Delta t\int_{\partial e} \phi \MM{u}\cdot\MM{n} 
\rho^* \diff s = 0,
\quad \forall \phi \in DG(n),
\end{equation}
where $\rho^*=\Pi_2\hat{\rho} = \Pi_2\Pi_1\rho^n$.
\end{enumerate}

We now develop the limiter $\Pi_2$.

If we choose $\phi$ as the indicator function for element $e$, \emph{i.e.},
\begin{equation}
\phi(\MM{x}) = \left \{
\begin{array}{rl}
1 & \mbox{ if } \MM{x}\in e, \\
0 & \mbox{ otherwise,} \\
\end{array}\right.
\end{equation}
then Equation \eqref{eq:forward euler rho} becomes
\begin{equation}
\int_e \rho^{n+1} \diff x 
= \int_e \rho^* \diff x 
- \Delta t\int_{\partial e} \MM{u}\cdot\MM{n} 
\rho^* \diff s = 0.
\end{equation}
This may be written in the form
\begin{equation}
\label{eq:rho n+1}
\overline{\rho}^{n+1} = \overline{\rho}^n\left(1 + c^- - c^+\right),
\end{equation}
where 
\begin{equation}
  c^{\pm} = \pm\frac{\Delta t}{|e|}
\int_{\partial e^{\pm}} \MM{u}\cdot\MM{n} 
\frac{\tilde{\rho}^*}{\overline{\rho}^n}\diff s,
\end{equation}
where $\partial e^+$, $\partial e^-$ are the parts of $\partial e$
where $\MM{u}\cdot\MM{n}$ is positive and negative respectively.  Both
$c^+$ and $c^-$ are positive if $\rho$ is positive everywhere. 

For $\overline{\rho}^{n+1}$ to be positive, we need
\begin{equation}
\label{eq:positivity}
c^+ < 1 + c^-.
\end{equation}
This will always be satisfied for sufficiently small $\Delta t$. 
However, it may be violated if there are dramatic jumps in $\rho$, even
for moderate values of the upwind/downwind Courant numbers
\begin{equation}
\tilde{c}^{\pm} = \pm\frac{\Delta t}{|e|}\int_{\partial e^\pm}
\MM{u}\cdot\MM{n} \diff s.
\end{equation}
Hence, we use a failsafe limiter to adjust the slope in each element
where Equation \eqref{eq:positivity} is not satisfied, to ensure
positivity.

Since $c^-$ depends on values of $\rho$ from outside element $e$,
it will not be altered by limiting the slope in element $e$. Hence,
if the positivity condition \eqref{eq:positivity} is not satisfied,
applying a limiter
\begin{equation}
\rho^* = \overline{\tilde{\rho}} + \beta\left(
\tilde{\rho} - \overline{\tilde{\rho}}
\right),
\end{equation}
where $0\le \beta \le 1$, results in replacing $c^+$ by
\begin{equation}
\tilde{c}^+ + \beta(c^+-\tilde{c}^+).
\end{equation}
Hence, the positivity condition becomes
\begin{equation}
\label{eq:limited positivity}
\tilde{c}^+ + \beta(c^+-\tilde{c}^+) < 1 + c^-.
\end{equation}
It is always possible to achieve this condition by setting
$\beta=0$, in which case the condition becomes
\begin{equation}
\tilde{c}^+ < 1 + c^-,
\end{equation}
for which the CFL condition
\begin{equation}
\tilde{c}^+ < 1,
\end{equation}
is sufficient. Hence, we choose $\beta\leq 1$ to be as
large as possible provided that Equation \eqref{eq:limited positivity}
is satisfied, \emph{i.e.},
\begin{equation}
\beta =
\left\{
\begin{array}{c r}
 \max\left(0,\min\left(1,\frac{1 + c^- - \tilde{c}^+}
{c^+-\tilde{c}^+}\right)\right)
\end{array}\right..
\end{equation}

\section{Reconstruction of flux}
If the advection equation for the tracer q wish to be solved provided the conti
nuity equation is solved for density $\rho$, the flux in the advection equation
needs to be reconstructed. following from the paper {NEED CITATION HERE}
but
\section{Bounded advection}

The forward Euler conservative DG discretisation for the transport
equation for $q$ is,
\begin{equation}
\int_e \phi \rho^{n+1}q^{n+1} \diff x 
= \int_e \phi \rho^nq^n \diff x 
+ \Delta t \int_e \nabla \phi \cdot\MM{F}^n q^n\diff x
- \Delta t\int_{\partial e} \phi \MM{F}^n\cdot\MM{n} \tilde{q}^n \diff s = 0,
\quad \forall \phi \in DG(n),
\end{equation}
where $\tilde{q}^n$ is the value of $q^n$ on the upwind side of
$\partial e$.

Since this linear scheme does not guarantee bounds on $q^{n+1}$, we
also apply limiters to ensure this, following the same two step
strategy, first ensuring that all values of $q$ lie between bounds
defined from element mean values, then adjusting the slope so that the
outflow values guarantee positive element mean values in the next
timestep. We use $\rho$-weighted element mean values defined by
\begin{equation}
\overline{q} = \frac{1}{\overline{\rho^n}}\int_e \rho^n q \diff x,
\end{equation}
so that
\begin{equation}
\overline{\rho^{n}}\overline{q} = \int_e \rho^n q \diff x.
\end{equation}

In this case, replacing $q$ by the limited function
\begin{equation}
\Pi_1q = \overline{q} + \alpha(q-\overline{q}),
\end{equation}
does not violate conservation, since
\begin{align}
\int_e \rho^n\Pi_1q \diff x &= \overline{q}\int_e \rho^n\diff x + \alpha 
\left(\int_e \rho^nq\diff x-\overline{q}\int_e \rho^n\diff x\right), \\
&= \overline{\rho^n}\overline{q} + \alpha\left(\overline{\rho^n}\overline{q}
-\overline{\rho^n}\overline{q}\right) = \overline{\rho^n}\overline{q},
\end{align}
as required.

The forward Euler timestep for $q$ is then as follows:
\begin{enumerate}
\item A pre-processing limiter applied to $q$, such as the
  Barth-Jespersen limiter or the Kuzmin vertex-based limiter, that
  sets bounds $(q_{\min},q_{\max})$ on each element $e$, based on the
  mean value $\overline{q}^n_e$, and the value in selected surrounding
  elements. The limiter is then defined by
\begin{equation}
\Pi_1q^n|_e = \overline{q}^n + \alpha (q^n - \overline{q}^n),
\end{equation}
with $0\leq \alpha \leq 1$ chosen as the maximum value such that
$\Pi q^n$ lies within the bounds in element $e$.

If the element mean values of $q^n$ are all positive, then so are
the bounds, and therefore $\Pi q^n$ is positive everywhere.
\item A second pre-processing limiter $\Pi_2$ to guarantee that the
  element means $\overline{q}^{n+1}$ will lie within their bounds at the
  following the Euler step. The goal of this section is to introduce
  such a limiter.
\item The application of the forward Euler step,
\begin{equation}
\label{eq:forward euler q}
\int_e \phi\rho^{n+1} q^{n+1} \diff x 
= \int_e \phi \rho^n q^* \diff x 
+ \Delta t \int_e \nabla \phi \cdot\MM{u}\rho^n q^*\diff x
- \Delta t\int_{\partial e} \phi \MM{u}\cdot\MM{n} 
\rho^n q^* \diff s = 0,
\quad \forall \phi \in DG(n),
\end{equation}
where $q^*=\Pi_2\hat{q} = \Pi_2\Pi_1q^n$.
\end{enumerate}

We now develop the limiter $\Pi_2$ for $q$, assuming that $q$ already
satisfies the bounds in each element.

If we choose $\phi$ as the indicator function for element $e$, then
Equation \eqref{eq:forward euler q} becomes
\begin{equation}
\int_e \rho^{n+1} q^{n+1}\diff x 
= \int_e \rho^n q^*\diff x 
- \Delta t\int_{\partial e} \MM{u}\cdot\MM{n} 
\rho^nq^* \diff s = 0.
\end{equation}
This may be written in the form
\begin{equation}
\overline{\rho}^{n+1}\overline{q^{n+1}}
= \overline{\rho}^n\left(q^n + c^-q^- - c^+q^+\right),
\end{equation}
where
\begin{align}
\overline{q^{n+1}} & = \frac{1}{\overline{\rho^{n+1}}}\int_e
\rho^{n+1}q^{n+1}\diff x, \\
q^{\pm} & = \pm\frac{1}{c^{\pm}}\int_{\partial e^{\pm}}
\MM{F}\cdot\MM{n}\tilde{q}\diff s,
\end{align}
and $c^\pm>0$ are defined as before. Finally, making use
of Equation \eqref{eq:rho n+1}, we obtain
\begin{equation}
(1+ c^- - c^+)\overline{q^{n+1}}
= \bar{q}^n + c^-q^- - c^+q^+,
\end{equation}
where positivity ensures that $1+c^--c^+>0$. We see that
$\overline{q}^{n+1}$ might exceed the bounds if $q^+$ is too far below
$\bar{q}^n$.

If our limiter scheme has an imposed maximum bound $\bar{q}^{n+1}
\leq q^{\max}$, we now need to adjust the slope in $q^n$,
\begin{equation}
q^n \mapsto \bar{q}^n + \beta(q^n - \bar{q}^n),
\end{equation}
where $0\leq \beta \leq 1$, in order to replace
\begin{equation}
q^+ \mapsto \bar{q}^n + \beta(q^+-\bar{q}^n).
\end{equation}
Since $q^-\leq q^{\max}$, since we have already limited the slope so
that $q$ satisfies the element bounds, which are the same on each side
of each element boundary. Therefore, if we choose $\beta=0$, the update 
becomes
\begin{equation}
(1+ c^- - c^+)\overline{q}^{n+1}
= \bar{q}^n(1-c^-) + c^-q^- < q^{\max}(1 + c^- - c^+),
\end{equation}
and the bound will be satisfied as required. We choose the maximum
value of $\beta\leq 1$ such that
\begin{equation}
\bar{q}^n(1-c^+) + c^-q^- - c^+\beta(q^+-\bar{q}^n)
\leq (1+ c^- - c^+)q^{\max},
\end{equation}
\emph{i.e.},
\begin{equation}
\beta = \min\left(1,
 \frac{(1+ c^- - c^+)q^{\max} 
- \left(\bar{q}^n(1-c^+) + c^-q^- \right)}
{c^+(\bar{q}^n-{q}^+)}\right).
\end{equation}

\bibliography{limiter_notes}

\end{document}
